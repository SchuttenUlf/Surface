\section{Ergebnisse}
\label{sec:ergebnisse}
\subsection{Berechnung des Apparatekorrekturfaktors}

Der Apparatekorrekturfaktor K$_{\text{Ka}}$ berechnet sich aus dem Verhältnis des theoretischen Wertes für de Oberflächenspannung des bidestillierten Wasers bei \SI{20}{\degreeCelsius} zum am Tensiometer abgelesenen Wert. Die Berechnung mit den für diesen Versuch gültigen Werten ist in Gleichung \eqref{Kkal} dargestellt. Der $abgelesene Wert$ stammt aus der Messreihe zur Kalibrierung bei \SI{20}{\degreeCelsius}, welche auf dem Angehängten Deckblatt zum Versuch einzusehen ist.

\begin{flalign}\label{Kkal}
	\text{K}_{\text{Ka}}&=\frac{theoretischer Wert}{abgelesener Wert}\\
	&=\frac{\SI{72,8}{\milli\newton\per\meter}}{\SI{70,0}{\milli\newton\per\meter}}\\
	&=\underline{\underline{1,04}}
\end{flalign}


\subsection{Einwaagen und Volumina der hergestellten Lösungen}

\begin{table}[h!]
	\centering
	\caption{Massen und Volumina der zur Herstellung der Lösungen verwendeten Chemikalien}
	\label{tab:Einwaagen}
	%\resizebox{12.6cm}{!}{
	\begin{tabular}{|c|c|c|c|}
		\hline
	\multirow{ 2}{*}{\textbf{Stoff}} & \textbf{Zielkonzentration } & \textbf{Einwaage}  & \textbf{Volumen} \\ 
		&[\si{\mole\per\liter}]&[\si{\gram}]&[\si{\milli\liter}]\\
		\hline
		Ethanol & 0,1 & - & 0,58\\ 
		Natriumchlorid& 0,1 & 0,58433 &-\\ 
		Natriumdodecylsulfat (\SI{10}{\gram\per\liter})& 0,001& -&0,28\\
		\hline
		
	\end{tabular}
	%	}
\end{table}
\FloatBarrier
\vspace*{-2.5mm}
%Tabelle Ende
\subsection{Korrektur der abgelesenen Oberflächenspannungen}

Die Korrektur der abgelesenen Oberflächenspannungen ($\sigma^\ast$) erfolgt durch die Multiplikation mit dem Apparatekorrekturfaktor (K$_{\text{Ka}}$) und dem Korrekturfaktor nach \textsc{Harkins \& Jordan} (K). Letzterer Korrekturfaktor wurde aus der, der Versuchsanleitung angehängten, Tabelle 01 entnommen. In den Fällen, wo ein Wert nicht passend eingetragen war, wurde der dem gesuchten Wert am nächsten liegende genutzt.

Die Korrekturen erfolgten analog der Rechnung in Gleichung \eqref{korrektur}.

\begin{flalign}\label{korrektur}
	\sigma&=\sigma^\ast*\text{K}_{\text{Ka}}*\text{K}\\
	&=\SI{69,8}{\milli\newton\per\meter}*1,04*0,996\\
	&=\underline{\underline{\SI{72,30}{\milli\newton\per\meter}}}
\end{flalign}

\subsection{Prüfung auf Ausreißer}

Das Kriterium für einen Ausreißer wurde als Abweichung um mehr als die dreifache Standardabweichung vom Mittelwert definiert ($\pm 3*\sigma$). Es wurde ein Ausreißer gefunden. Es handelt sich um den dritten Messwert zum Einfluss des Geschirrspülmittels. Dieser wurde daher gestrichen.

\subsection{Probe auf linearen Zusammenhang}

Durch auftragen der gemessenen und korrigierten Oberflächenspannungen des destillierten Wassers über der Temperatur wird ein Graph erzeugt, welcher es erlaubt, eine Aussage über die Linearität der Messpunkte, zu treffen.

\begin{figure}[h!]
	\begin{center}
		\resizebox{0.8\textwidth}{!}{
			\begin{tikzpicture}[trim axis left, trim axis right]
			\begin{axis}[
			axis lines = left,
			width = 15cm,
			height = 11cm,
			xmin = 0,
			xmax = 67,
			ymin = 66,
			ymax = 74.5,
				ytick = {50,51,...,74},
				xtick = {0,5,...,65},
			ylabel={Oberflächenspannung in \si{\milli \newton \per \meter}},
			%y label style={at={(0,0.5)}},
			xlabel={Temperatur in \si{\celsius}},
			legend style={at={(0.35,0.9)},anchor=west},
			%	y dir = reverse,
			]				
			\addplot [color=black, mark=*] coordinates{(20,72.5) (30,72.198)  (40,70.41) (60,67.19) };
			
			\addplot +[mark=none, dashed, black, domain=15:65] {-0.1400728183*x + 75.830567};
			
			
			
			\legend{gemittelte und korrigierte Oberflächenspannungen, Regressionsgerade $\sigma=-\SI{0,1400}{}*T + \SI{75.83}{} \, | \, R^2$ = \SI{0,962}{}}
			\end{axis}
			\end{tikzpicture}}
		\caption{Oberflächenspannung des Wassers in Abhängigkeit der Temperatur}
		\label{dia:oberflaechenT}
	\end{center}
\end{figure}
\FloatBarrier


Diskussion- ersten beiden selbst aufgenommen, schlecht kalibriert, letzte Beiden gegeben. Korrektur-kalibrierungsfehler steht schon in Fehlerbetrachtung.