\newpage
\section{Fehlerbetrachtung}
\label{sec:fehler}
Die Ableseungenauigkeit wurde im Kapitel \ref{sec:durchfuerung} bereits angesprochen. Die Nachkommastellen unterliegen daher einer geschätzten Abweichung von $\pm$\SI{0,2}{\milli\newton\per\meter}. Im laufe des Versuches wurde die Apparatur, durch auflegen der Handballen beim vorsichtigen Hantieren an den Stellschrauben, zwei  mal zum Kippeln gebracht. Dabei könnten sich die Einstellungen geringfügig verändert haben. Es waren aber keine Abweichungen festzustellen. Nach jedem Probenwechsel muss einige Zeit gewartet werden, bis der Temperiermantel die Probe gleichmäßig auf die am Thermostat eingestellte Temperatur gebracht hat. Die Einstellung des thermischen Gleichgewichts konnte nur geschätzt werden. Es ist wahrscheinlich, dass mit den Messungen begonnen wurde, bevor das Medium die gewünschte Temperatur hatte. Zur Herstellung der Natriumdodecylsulfatlösung wurde der falsche Ausgangsstoff verwendet. Das Natriumdodecylsulfat setzt sich mit der Zeit am Boden des Gefäßes ab. Aus diesem Grund befindet sich stets eine kleinere Menge auf dem Laborschüttler. In diesem Falle stammte die Reagenz aber aus der Vorratsflasche. Diese wurde nur kurz aufgeschüttelt. Es muss angenommen werden, dass die erhaltene Lösung eine etwas geringere Konzentration aufwies. Des weiteren wurde im Zusammenhang mit den Natriumdodecylsulfat festgestellt, dass verhältnismäßig große Abweichungen zwischen den einzelnen Messergebnissen auftraten. Als entscheidende Einflussgröße auf das Abrissverhalten wurde die Hebegeschwindigkeit des Ringes ermittelt. Je langsamer die Spannung gesteigert wurde, desto geringer war die gemessene Kraft, bei der der Film abriss. Die rundheit des Ringes wirkt sich auf die Messergebnisse aus. Es ist zu beachten, dass der Ring nie ideal kreisförmig sein wird, auch wenn durch Sichtprüfung keine Formfehler festgestellt werden konnten. Im Probengefäß kommen auch Lösungen zum Einsatz. Rückstände selbiger könnten trotz des gründlichen Spülens die nachfolgenden Messungen beeinflusst haben. Die verwendeten Korrekturfaktoren nach \textsc{Harkins \& Jordan} sind aus einer Tabelle abgelesen und nur für die jeweiligen Messwerte angenähert. Die Beschreibung selbiger durch eine Funktion, hätte vermutlich zu präziseren Ergebnissen geführt. Zur Herstellung der Lösungen, hätte die Dosierung der Volumina, durch zusätzliches Einwiegen der Flüssigkeiten, abgesichert werden können. Die Positionierung des Ringes im Probengefäß wirkt sich auch auf das Versuchsergebnis aus. In den Randbereichen unterliegt der Ring auch Wechselwirkungen mit dem Meniskus an der Gefäßwand. Wäre das Gefäß größer dimensioniert, könnte diesem Einfluss besser entgegengewirkt werden.
Die Kalibrierung oder die Korrektur der Messwerte muss fehlerbehaftet sein, weil die korrigierte Oberflächenspannung nicht auf den Erwartungswert von \SI{72,8}{\milli\newton\per\meter} kommt. Es stellt sich heraus dass ein Korrekturfaktor K von 1, anstatt der aus der Tabelle abgelesenen 0,996, zum gewünschten Ergebnis führen würde. Diese Änderung verbessert außerdem den Bestimmtheitsgrad der Regressionsgerade im Kapitel \ref{sec:ergebnisse} 
