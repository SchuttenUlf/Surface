\section{Einleitung und Versuchsziel}
\label{sec:aufgabenstellung}
%In der Aufgabenstellung wird (in eigenen Worten und ganzen Sätzen) formuliert, was das Ziel des 
%Versuches ist.  
%[Beachten Sie die eigentliche Aufgabenstellung in den Versuchsanleitungen sowie die Hinweise zur Auswertung!] 
Die Oberflächenspannung ist ein allgegenwärtiges Phänomen. Sie ist auch in Industrie und Technik von großer Bedeutung, da  durch sie Grenzflächen und ihr Verhalten geprägt werden. Sie ist essentiell, um Prozesse mit Phasenübergängen, das Benetzungsverhalten und die Tropfenbildung zu beschreiben.

In diesem Versuch wurden Untersuchungen zum Verhalten der Oberflächenspannung von Flüssigkeiten unter dem Einfluss von Temperatur und Fremdstoffbeimischungen durchgeführt. 

\section{Theoretische Grundlagen}

Zwischen den Teilchen einer Flüssigkeit wirken Kohäsions- und Adhäsionskräfte. Im Inneren einer Flüssigkeit erfährt jedes Teilchen, von jeder Seite, die gleichen Anziehungs- und Abstoßungskräfte. An der Grenzfläche der Flüssigkeit hin zu einem Gas sieht sich das Flüssigkeitsteilchen unterschiedlichen Abstoßungs- und Anziehungskräften ausgesetzt. Die Anziehungskräfte zwischen den Flüssigkeitsmolekülen wirken dabei stärker als die zu den Gasmolekülen. Es resultiert eine senkrecht auf die Flüssigkeitsoberfläche gerichtete Kraft. Um diese zu überwinden und die Flüssigkeitsoberfläche zu vergrößern, muss eine entsprechend größere Kraft bzw. Arbeit aufgebracht werden. Dabei ist die aufzuwendende Arbeit proportional zur betrachteten Oberflächenänderung (siehe Gl.\eqref{arbeit}). Als Proportionalitätsfaktor wird die Oberflächenspannung $\sigma$ eingeführt. Diese ist stoff- und temperaturabhängig. 

\begin{equation}\label{arbeit}
	\Delta W=\sigma*\Delta A
\end{equation}
Es existieren einige Möglichkeiten die Oberflächenspannung zu bestimmen. Hier wird die Oberflächenspannung mit der Ringmethode nach \textsc{Du Noüy} gemessen. Dabei wird ein Platin-Iridium-Ring aus einer Flüssigkeit herausgezogen. Aus der Kraft, die aufgebracht werden muss um den anhaftenden Flüssigkeitsfilm zum Abriss zu bringen, geht die Oberflächenspannung hervor. Die Kraftmessung erfolgt durch eine Torsionswaage mit einem Waagebalken, an welchem der Ring befestigt ist (vgl. Abb.\ref{fig:versuchsaufbau-ebull}). Der erhaltene Messwert muss anschließend korrigiert werden. Dabei werden die Korrekturfaktoren die Einflüsse der Geometrie des Ringes, des angehobenen Flüssigkeitsfilmes und des Apparates berücksichtigt. Die vollständige Formel ist in Gleichung aufgeführt.

\begin{equation}
	\sigma=\sigma^\ast*\text{K}_{\text{Ka}}*\text{K}
\end{equation}

Die statistischen Berechnungen zur Versuchsauswertung basieren auf dem Mittelwert und der Standardabweichung entsprechend der Formeln \eqref{Gl:Mittelwert} und \eqref{Gl:standardabweichung}.

\begin{flalign}
\label{Gl:Mittelwert}
\bar{x} &= \frac{\sum_{n=1}^{N}x_n}{N}
\end{flalign}


\begin{flalign}\label{Gl:standardabweichung}
s &= \sqrt{\frac{\sum_{n=1}^{N}(x_n-\bar{x})^2}{N-1}}
\end{flalign}

Ein Ausreißer ist ein Messwert der überdurchschnittlich stark vom Rest der Messwerte Abweicht, und das Endergebnis dadurch verfälscht. In diesem Fall werden Messwerte dann als Ausreißer angenommen, wenn sie um die dreifache Standardabweichung von berechneten Mittelwert abweichen. Die unteren und oberen Vertrauensgrenzen berechnen sich dann entsprechend der Formeln \eqref{gl:ausu} und \eqref{gl:auso}.

\begin{equation}\label{gl:ausu}
\text{untere Vertrauensgrenze }=\bar{x} - 3*s
\end{equation}

\begin{equation}\label{gl:auso}
\text{obere Vertrauensgrenze} =\bar{x} + 3*s
\end{equation}

