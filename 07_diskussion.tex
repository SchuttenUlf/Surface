\section{Diskussion der Ergebnisse}
\label{sec:diskussion}
Aus der Abbildung \ref{dia:oberflaechenT} ist eindeutig zu ersehen, dass die OberflächenSpannung mit steigender Temperatur linear abnimmt. Der Bestimmtheitsgrad der Regressionsgerade von 0,962 wird als hinreichend hoch erachtet. Eine Ursache für die Abweichungen zwischen dem ersten und dem zweiten Teil der Messpunkte, liegt darin, dass jeweils andere Messreihen, aus zu verschiedenen Zeiten durchgeführten Messungen, zugrunde liegen.




Der Vergleich der Oberflächenspannungen in Tabelle \ref{tab:vergleich} ergibt, dass alle Beimischungen von Fremdstoffen die Oberflächenspannung des Wassers verringern. Die Kennzeichnung mit dem Asterisk weist darauf hin, dass nicht der Reinstoff untersucht, sondern jener Stoff dem Wasser untergemischt wurde. Die Senkung der Oberflächenspannung beruht dabei auf der Störung der Ausbildung von Wasserstoffbrückenbindungen zwischen den Wassermolekülen. Der Vergleichsweise Geringe Effekt des Natriumchlorids ist dadurch zu begründen, dass es sich dabei um ein relativ kleines, polares Molekül handelt, welches sich vergleichsweise gut in das "`Gitter"' der Wassermoleküle einfügt. Ethanol ist dagegen weniger polar und zudem auch noch deutlich größer als das Wasser. Das führt zu einer noch stärkeren Störung der Wasserstoffbrückenbindungen. 
Das Natriumdodecylsulfat und das fit-Spülmittel repräsentieren die Gattung der Tenside. Diese zeichnen sich durch eine lange Kohlenstoffkette aus, welche durch ihre Unpolarität hydrophob wirkt. Der hydrophobe Teil der Moleküle ist daher bestrebt sich in Kontakt mit der Luft zu begeben und sich so an der Oberfläche anzureichern. Damit werden die bindenden Wechselwirkungen zwischen den oberflächennahen Wassermolekülen entscheidend gestört, was sich in einer massiven Verringerung der Oberflächenspannung niederschlägt. Diese wird, wie auch in Tabelle \ref{tab:vergleich} zu erkennen ist quasi halbiert.
\vspace{3mm}
Die aus dem gefundenen linearen Zusammenhang hervorgehende Oberflächenspannung des Wassers bei \SI{90}{\degreeCelsius} von \SI{63,23}{\milli\newton\per\meter} weicht nur relativ wenig vom Literaturwert\footnote{Ulrich Grigull, Johannes Staub, Peter Schiebener (1990): Steam Tables in SI-Units - Wasserdampftafeln. Springer-Verlagdima Gmbh } \SI{60,82}{\milli\newton\per\meter} ab. Es ist dabei zu beachten, dass zurFormulierung der verwendten Geradengleichung nur 4 Datenpunkte Verwendung fanden und Daten aus verschiedenen Messreihen kombiniert wurden. Die abweichung betragt knapp 4\%. 
\vspace{3mm}
Die Messdaten sind recht präzise. Die vermuteten Wirkungen auf die Grenzflächen konnten bestätigt werden. Es ergab sich nur ein Messwert als Ausreißer. Auch die Abweichung der berechneten Oberflächenspannung für Wasser bei \SI{90}{\degreeCelsius} von nur 4\% spricht für die Glaubwürdigkeit des Datensatzes.

\begin{table}[]
	\centering
	\caption{Der Größe nach sortierte Mittelwerte der gemessenen Oberflächenspannungen bei \SI{20}{\degreeCelsius}}
	\label{tab:vergleich}
	\resizebox{12.6cm}{!}{
		\begin{tabular}{|c|c|}
			\hline
			\multirow{ 2}{*}{\textbf{Stoff}} & \textbf{ Oberflächenspannung } \\ 
			&[\si{\milli\newton\per\meter}]\\
			\hline
			reines Wasser&72,509\\
			Natriumchlorid*& 72,343 \\ 
			Ethanol* & 71,394 \\ 
			Natriumdodecylsulfat*& 44,180\\
			fit-Spülmittel*&33,173\\
			\hline
			
			\hline
		\end{tabular}
	}
\end{table}
\FloatBarrier

%Tabelle Ende